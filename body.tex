% put environments that should be ignored by texcount here, e.g., here listing for code

%TC:envir listing [] ignore

%for reference to this section
\section{Introduction}
\label{section:Introduction}

\textcite[]{kerr_lerf_2023} Test Citation 1

This paper explores the historical evolution of the rapidly evolving field of 3D mesh reconstruction techniques as well as its ethical implications and applications.
Beginning with an overview of the historical development of these techniques, ranging from early methodologies to modern advancements, covering significant milestones that have shaped the entire subject.

The examination of the applications of 3D mesh reconstruction across diverse domains, including development and entertainment, highlights the potential for innovation and advancement. However, it also raises crucial ethical concerns. Privacy considerations, issues of representation and cultural sensitivity, as well as the implications of estimations and hallucinations, highlight the complex ethical landscape surrounding these technologies.


This template is used for seminar papers, bachelor and master theses at MultiMediaTechnology of the Salzburg University of Applied Sciences.

The structure of the template fits many theses works. Seminar papers often require their own structure as it is a literature review on a specific topic and does not present your own work.

Outline the research field and lead towards your research question. How is the investigated issue resolved in related work? What are limitations of these solutions? What is your contribution to find a solution?

\section{Related Work}
Introduce why this specific related work is important for your own work. Which areas do you cover and why? What do you take as inspiration and what do you do differently/improve upon?

% \section{System Overview}
% Provide a high level overview of your system, approach, etc.
% Describe features, user interfaces, provide screenshots.
% What does a user do with your application/system/interaction method?

% \begin{listing}[H]
%     \begin{csharpcode}
%     for (var item in myList)
%     {
%         Console.WriteLine($"Fancy syntax highlighting for {item}");
%     }
%     \end{csharpcode}
%     \caption{Example of a listing.}
%     \label{lst:example}
% \end{listing}

% See code \ref{lst:example} for high efficiency code.

\section{Historical Evolution}

\subsection{Models}

\subsection{Comparison}

\section{Applications of 3D Mesh Reconstruction}

\subsection{Development}

\subsection{Entertainment}

\subsection{Medical}

\subsection{Other Applications}

\subsubsection{Cultural Heritage}

\section{Ethical Implications}

\subsection{Environmental Impact}

\subsection{Privacy Concerns}

\subsection{Cultural Sensitivity}

\subsection{Implications of Estimations and Hallucinations}

% \section{Implementation}
% Provide implementation details such as the used software and our software architecture, highlight your own solutions to encountered difficulties. Describe relevant iterations of your implementation.

\section{Evaluation}
Describe your methodology. How did you evaluate your work? Why did you choose this methodology? Present results of your evaluation here.

\section{Discussion and Future Direction}
Discuss your results to answer your research question. Does your data support you hypotheses? Put your results into perspective by situating it in the research field/related work.

\section{Conclusion}
Summarize your work, outline limitations and future work.

% \section{Formatierung}
% \label{section:Formatting}

% Text mit beliebigen Sonderzeichen in UTF-8 ohne BOM \ldots
% ,
% \textbf{hervorgehobener Text},
% \texttt{void}\footnote{Fußnote 1},
% mathematische Formel im Text $\sum_{i=0}^n i^2$
% \ldots

% Referenz auf Unterabschnitt \ref{subsection:Coding} der Arbeit, automatisch richtig nummeriert.

% \textcite[]{Mulloni:2010} für einen einen Literaturverweis im laufenden Text.

% Literaturverweise sind essentiell für eine wissenschafliche Arbeit. \autocite[]{McConnell:2004:CCS:1096143}.

% Achtung: nur zitierte Literatur wird im Literaturverzeichnis
% angeführt.\footnote{Fußnote 2}


% Wir verwenden \LaTeX\footnote{ \url{http://en.wikibooks.org/wiki/LaTeX}} -- und das
% ist keine Quelle, sondern blos eine URL.

% \subsection{Figures machen was sie wollen}

% % h = try to place the figure Here
% % t = try to place the figure at the Top of a page
% % p = try to place this figure along with others on a separate Page
% % Note that LaTeX has a sophisticated ranking algorithm to place figures.
% % It is not always easy to accept LaTeX's placing but it is harder doing it
% % manually. Just let it go ;-)
% \begin{figure}[!ht]
% 	\centering
% 	\subfloat[Das Julia Fraktal]{
% 		\includegraphics[width=0.75\linewidth]{images/Julia-Fractal.png}
% 		%for reference of this subfigure only
% 		\label{subfigure:Julia-Fractal}
% 	}
% 	\qquad
% 	\subfloat[Noise für Tinteneffekte]{
% 		\includegraphics[width=0.75\linewidth]{images/Perlin-Coherent.png}
% 		%for reference of this subfigure only
% 		\label{subfigure:Perlin-Coherent}
% 	}
% 	\caption[
% 		Verschiedene Pixelgraphiken\newline
% 		% source url given in the table of figures
% 		\small\texttt{https://mediacube.at/wiki/}
% 	]{
% 		Verschiedene Pixelgraphiken
% 	}
% 	%for reference to all subfigures
% 	\label{figure:PixelImages}
% \end{figure}

% Unterstützte Pixelgraphikformate: PNG, JPEG, PDF.
% Angabe von height oder width meist wichtig.

% Referenz auf Abbildung \ref{figure:PixelImages} mit allen Teilbildern.
% Referenz auf Unterabbildung \ref{subfigure:Julia-Fractal}.

% %figure* stretches figure over both columns
% \begin{figure*}[t]
% 	\centering
% 	\includegraphics[width=0.9\textwidth]{images/KappaGamma.pdf}
% 	\caption{
% 		Vektorgraphik mit \LaTeX\ Beschriftung ($\kappa$, $\gamma$)
% 	}
% 	%for reference to this figure
% 	\label{figure:KappaGammaTau}
% \end{figure*}

% Referenz auf Abbildung \ref{figure:KappaGammaTau}.

% Bei Vektorgraphik mit \LaTeX\ Beschriftung keine Skalierung mit width oder height verwenden.

% Vektorgraphik mit \LaTeX\ Beschriftung kann etwa mit \texttt{ipe} erstellt werden.

% Unterstütztes Vektorgraphikformat: PDF. EPS muss konvertiert werden.


% \subsection{Unterabschnitt 2}
% %for references to this subsection
% \label{subsection:Coding}

% \begin{listing}[H]
%     \begin{csharpcode*}{firstnumber=10}
%         while (true)
%         {
%             // Ignition
%         }
%     \end{csharpcode*}

%     \caption{Example of another listing.}
%     \label{lst:Main}
% \end{listing}

% Wie man in Listing \ref{lst:Main}, kann man die erste Zeilennummern im Listing absichtlich ändern, hier z.B. auf 10. Beachte, dass man hier chsarpcode* als Umgebung nutzt, um neben
% den Default-Settings zusätzliche Einstellungen zu tätigen.

% \subsubsection{Unterunterabschnitt i}

% Wörtliches Zitat:
% %select proper language if not in German
% \selectlanguage{english}
% \begin{quote}
% ``Erwin Unruh discovered that templates can be used to compute
% something at compile time. [...] The intriguing part of this exercise, however, was that the production of the prime numbers was performed by the compiler during the compilation process and not at run time.''

% \autocite[305]{Bosch2014}
% \end{quote}
% %select German again or the language that you were using before (note ngerman stands for New German)
% %\selectlanguage{ngerman}
% \selectthesislanguage


% \subsection{Unterabschnitt b}

% \begin{enumerate}
% 	\item Punkt 1
% 	\begin{enumerate}
% 		\item Unterpunkt 1
% 		\item Unterpunkt 2
% 	\end{enumerate}
% 	\item Punkt 2
% \end{enumerate}

% \begin{itemize}
% 	\item Punkt 1
% 	\begin{itemize}
% 		\item Unterpunkt 1
% 		\item Unterpunkt 2
% 	\end{itemize}
% 	\item Punkt 2
% \end{itemize}


% \subsection{Unterabschnitt c}

% \begin{table}[ht]
% 	\centering
% 	\begin{tabular}{r|rrr}
% 		    & $i$ & $j$ & $k$ \\ \hline
% 		$i$ &$-1$ & $k$ &$-j$ \\
% 		$j$ &$-k$ &$-1$ & $i$ \\
% 		$k$ & $j$ &$-i$ &$-1$
% 	\end{tabular}
% 	\caption{
% 		Multiplikationstabelle für Quaternionen
% 	}
% 	\label{table:Quaternions}
% \end{table}

% Referenz auf Tabelle \ref{table:Quaternions}.

% \section{Abschnitt 2}
% \label{section:MathematicalStuff}

% Sei $f(x)$ eine stetige Funktion, so ist die \textbf{Fourier Transformierte}
% $F(\omega)$ wie folgt definiert:
% \begin{equation}
% \label{equation:FourierDefinition}
% 	F(\omega) = \int_{-\infty}^{\infty} f(x) e^{-i\omega t} dt
% \end{equation}

% Referenz auf mathematische Gleichung (\ref{equation:FourierDefinition}).

% Unnummerierte Gleichung:
% \begin{equation*}
% 	e^{i\varphi} = \cos\varphi + i \sin\varphi
% \end{equation*}
% %you may also use \[ \] instead of \begin{equation*} and \end{equation*}

% Gleichungssystem:
% \begin{eqnarray}
% 	g(x) = f(x - x_0) & \Leftrightarrow &
% 		G(\omega) = F(\omega) e^{-i\omega x_0} \\
% 	g(x) = f(x) e^{i\omega_0 x} & \Leftrightarrow &
% 		G(\omega) = F(\omega - \omega_0)
% \end{eqnarray}
