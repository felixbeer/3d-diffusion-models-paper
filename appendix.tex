\begin{appendices}


\section{AI Methodology}

\textbf{You can delete the explanatory text in this section and add your AI documentation.}

Document in this section, how AI was utilized to support the creation of this seminar paper or thesis. Which tools were used? Which steps of the thesis creation were supported (e.g., brain storming, literature search, quality analysis of papers, text generation, text summaries, paraphrasing, spellchecking, ...). If generative AI was utilized (e.g., ChatGPT, CoPilot), also provide key text prompts that influenced the thesis.

In the following, list each tool and provide a brief description how the tool as utilized. For each tool, estimate in \% of the word count of the thesis, how much text was generated/modified/etc. by the AI tool. For the practical part of the thesis, estimate the \% of the results (e.g., code) was generated/mofidied/etc. by the AI tool. If a generative AI was used, provide exemplary prompts.

There is no requirement to highlight text in the thesis.

This section should have a maximum length of 2 pages.

\subsection{DeepL} % example tool
I used the tool (\url{https://addressoftool.com}) to [paraphrase|translate|...] text for this work.  I estimate that XX \% of the text was paraphrased with this tool.

\subsection{Copilot}
I used Copilot (\url{https://addressoftool.com}) to learn how to utilize RUST as a new programming language. ....
I estimate taht XXX \% of the source code was created with this tool.

\subsection{AI Tool 2}
... document tool usage ...

\subsection{Prompts}
I utilized the generative AI \url{https://bot.com} to create drafts for texts, support brainstorming, ...

... list exemplary prompts here ...

%\renewcommand{\thesubsection}{\Alph{subsection}}

% DIESEN TEIL NICHT LÖSCHEN ODER ÄNDERN == BEGINN
%\fi % end if for the if \ifmmtreviewversion
\end{appendices}
% DIESEN TEIL NICHT LÖSCHEN ODER ÄNDERN == ENDE
